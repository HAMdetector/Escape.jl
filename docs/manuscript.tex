%----------------------------------------------------------------------------------------
%	PACKAGES AND OTHER DOCUMENT CONFIGURATIONS
%----------------------------------------------------------------------------------------

\documentclass[fleqn,11pt]{SelfArx} % Document font size and equations flushed left

\usepackage{parskip}
\usepackage{microtype}
\usepackage{svg}
\usepackage{float}
\usepackage{placeins}
\usepackage[english]{babel} % Specify a different language here - english by default
\usepackage{listings}
\usepackage{fontspec}
\usepackage{hyperref}
% \usepackage{lipsum} % Required to insert dummy text. To be removed otherwise
\setmainfont{TeX Gyre Termes}
%----------------------------------------------------------------------------------------
%	COLUMNS
%----------------------------------------------------------------------------------------

\setlength{\columnsep}{0.55cm} % Distance between the two columns of text
\setlength{\fboxrule}{1pt} % Width of the border around the abstract

%----------------------------------------------------------------------------------------
%	COLORS
%----------------------------------------------------------------------------------------

\definecolor{color1}{RGB}{124,0,0} % Color of the article title and sections
\definecolor{color2}{RGB}{150,150,150} % Color of the boxes behind the abstract and headings

%----------------------------------------------------------------------------------------
%	HYPERLINKS
%----------------------------------------------------------------------------------------

\usepackage{hyperref} % Required for hyperlinks
\hypersetup{hidelinks,colorlinks,breaklinks=true,urlcolor=color2,citecolor=color1,
  linkcolor=color1,bookmarksopen=false,pdftitle={Title},pdfauthor={Author}}

%----------------------------------------------------------------------------------------
%	ARTICLE INFORMATION
%----------------------------------------------------------------------------------------

\PaperTitle{
    HAMDetector: Combining information to predict HLA-associated mutations with a
    Bayesian regression model
} % Article title

\Authors{Daniel Habermann\textsuperscript{1}, Daniel Hoffmann\textsuperscript{1}} % Authors
\affiliation{\textsuperscript{1}\textit{Bioinformatics \& Computational Biophysics, Faculty of Biology, University of Duisburg-Essen, 45117 Essen, Germany}} % Author affiliation
% \affiliation{\textsuperscript{2}\textit{Department of Chemistry, University of Examples, London, United Kingdom}} % Author affiliation
% \affiliation{*\textbf{Corresponding author}: john@smith.com} % Corresponding author

\Keywords{human leucocyte antigen system, HLA, multiple sequence alignment, escape mutations, 
  viral escape, Bayesian inference, sparsity, horseshoe, epitope prediction} % Keywords - if you don't want any simply remove all the text between the curly brackets
\newcommand{\keywordname}{Keywords} % Defines the keywords heading name

%----------------------------------------------------------------------------------------
%	ABSTRACT
%----------------------------------------------------------------------------------------

\Abstract{
  \\
  \textbf{Motivation} \\
  The human leucocyte antigen system (HLA) is of paramount importance to combat viral
  infections by presenting peptides on the cell surface via MHC I.
  In this way, CD8+ cytotoxic T-Lymphocytes exert a strong selection pressure towards
  virus variants that escape that immune recognition pathway, e.g. through point 
  mutations that decrease binding of the respective peptide to MHC I. \\

  Reliably identifying HLA-associated mutations is important for understanding viral
  evolution, but experimental methods like binding assays are prohibitively expensive
  for large-scale use and fail to recognize other mechanisms of immune escape like
  proteasomal processing.

  One step in finding these mutations is through the statistical analysis
  of sequence data. However, existing methods are based on nullhypothesis significance 
  testing and do not make use of all the available information and therefore have
  unsatisfactory real-world performance. \\

  \textbf{Results} \\
  Here, we present a Bayesian regression model that is easily extensible to include 
  information from different sources (e.g. epitope prediction software) 
  and makes use of recent advances in Bayesian inference, e.g. by using a sparsifying 
  prior. We show that including this kind of information improves predictive performance 
  considerably over state-of-the-art methods. \\

  \textbf{Availability and Implementation} \\
  The source code of this software is available at 
  \url{http://gogs.uni-due.de/habermann/Escape.git} under a permissive MIT license. \\

  \textbf{Supplementary information} \\
  \href{https://google.com}{Supplementary data} are available at \textit{Bioinformatics} online. \\
}

%----------------------------------------------------------------------------------------
\begin{document}

\flushbottom % Makes all text pages the same height

\maketitle % Print the title and abstract box

{
  \hypersetup{linkcolor=black}
  \tableofcontents
}

\pagebreak

\section{Introduction}

% - subset of peptidome presented on the cell surface via MHC I.
% - Cytotoxic t cells are selected to not recognize self.
% - by rule of exclusion, recognized peptides are foreign
% - t cells induce cytolytic activity and recruit other immune cells, reference

% - basic function of statistical analysis of sequence data
% - state of the art + brumme paper, phylogeny, linkage disequilibrium and codon covariation
% cite brumme and other paper

% - problems of p value based methods and statistical significance testing
% - highlight benefits of Bayesian approaches
\subsection{The HLA system}

One way how the human immune system is able to recognize intracellular viral infections
is through the human leucocyte antigen system:
In cells with active protein biosynthesis, proteins are continuously synthesized
and also degraded by a process called proteasomal degradation, which cleaves
proteins into linear peptides of varying length.
A small subset of these peptides is presented on the cell surface via
receptors called MHC class I. The genomic region encoding for MHC I is known to be
highly polymorphic, with more than 20000 different HLA alleles described today.
The resulting gene products differ in their binding properties, which means that
cells from different individuals present a highly diverse set of peptides on their
surface.
Cytotoxic T cells are selected during maturation to only weakly bind to 
peptide/MHC I complexes when the peptide originated from proteins of the usual proteome, 
but might be able to strongly bind to complexes of MHC I with peptides which are 
generated from of a viral protein.
Upon activation, T cells induce cytolytic activity and recruit other immune cells.

\subsection{HLA escape}
In this way, the HLA system exerts a strong selection pressure towards virus variants
that escape T cell recognition, for example through a point mutation
that results in reduced binding of an immunogenic peptide to MHC I or through a
set of mutations that alters the viral protein in such a way that it is cleaved into 
different peptides that are not recognized by the host's T cell repertoire.

The evolutionary events are complex and occur not only on the level of individuals, 
where a virus adapts to specific features of the host, but also on the population level,
because HLA alleles differ in their frequency across geographic regions, as they are inherited
according to standard Mendelian rules.

Upon transmission, HLA escape mutations typically quickly revert to their wild type
as they usually reduce viral replicative capacity (if a mutation would increase
viral replicative capacity regardless of the presence of a given HLA allele, it would
probably already be the wild type). Kawashima et al.
describe an escape mutation that is selected by HLA allele B\*53,
does not strongly affect viral replicative capacity and therefore slowly enriches
over time in Japan, where B\*53 commonly occurs.
How quickly a given escape mutation is selected upon transmission in a host depends on the
magnitude of the reduction in viral replicative capacity, on the strength of selection
pressure and also on the genetic background, e.g. some escape mutations require compensatory 
mutations which partly attenuate the negative impact on viral replicative capacity.

Studying HLA escape therefore provides an unique opportunity to gain insight into
viral evolution, on the host level but also on the population level.
Unfortunately, identifying HLA escape mutations is difficult in practice.

\subsection{Identifying HLA-escape mutations}

There are several experimental methods available to study HLA escape:
Recombinant MHC-I molecules can be used in binding assays: Upon complex formation
with a peptide, a change in conformation can be detected with conformation-specific
antibodies. This method is relatively fast, but only measures binding affinity of a
peptide to MHC-I and does not account for antigen processing or immunodominance, 
which describes the observation that a peptide may be presented via MHC-I on the cell surface but does
not induce an immune response.
An experimental setup that resembles the conditions in-vitro more closely but is also
more time-consuming is to measure
CD8+ T cell responses instead. This is usually done by stimulating PBMCs with prototype
and variant peptides and measuring the secretion of IFN-gamma by intracellular cytokine
staining and fluorescence-assisted cell sorting.
To analyse CD8+ T cell responses against endogenously processed antigens it is necessary
to generate cell-lines stably expressing the antigen in question and adding antigen-specific
CD8+ T cells. This method scales poorly as it requires transfection of cell lines and
antigen-specific expansion of CD8+ T cells.

\section{System and methods}
\section{Algorithm}
\section{Implementation}
\section{Discussion}
\section{References}

\thispagestyle{empty} % Removes page numbering from the first page
%----------------------------------------------------------------------------------------
%	ARTICLE CONTENTS
%----------------------------------------------------------------------------------------


\FloatBarrier

%----------------------------------------------------------------------------------------
%	REFERENCE LIST
%----------------------------------------------------------------------------------------
\phantomsection
\bibliographystyle{unsrt}

%----------------------------------------------------------------------------------------

\end{document}